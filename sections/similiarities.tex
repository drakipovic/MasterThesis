\chapter{Određivanje sličnosti izvornih kodova}

Ne postoji sustav specijaliziran za detekciju plagijata koji može sa sto postotnom sigurnošću utvrditi da je nešto plagijat. Kako bi odredili plagijat potreban je ljudski faktor.  Odmah možemo uočiti da to nije baš uvijek efikasno, kada bi morali pronaći plagijate među tisućama dokumenata čovjeku bi trebalo puno vremena. Upravo iz tog razloga razvijamo sustav koji bi odredio sličnost među parovima dokumenata te izbacio parove za koje smatra da nikako ne mogu biti plagijat te uvelike ubrzao i olakšao posao ljudima. Dokumenti mogu biti teksutalne datoteke, izvorni kodovi, pjesme, itd. \\
	
	Određivanje sličnosti izvornih kodova u svrhu detekcije plagijata je relativno neistraženo područje. Postoje dva vrlo slična, ali sada već stara sustava (nastali su prije više od 10 godina op.a)  \cite{moss} \cite{jplag} koji se baziraju na računanju otiska(eng. $fingerprint$) izvornog koda koji je detaljnije opisan u \cite{winnowing}. Razvijeni sustav nazvan $Turtle$ idejno je vrlo sličan sustavu razvijenom na završnom radu autora \cite{plagijator} koji je koristio algoritam detaljnije opisan u \cite{dorian}. $Turtle$ kao osnovni algoritam za računanje otiska te sličnosti izvornog koda također koristi algoritme iz \cite{winnowing}, no ideja se modificira i nadograđuje. U radu donosim još neka nova poboljšanja kao npr. veći broj operacija nad kodom u predprocesu, što rezultira boljim sličnostima među parovima te brže izvršavanje algoritma. U nastavku poglavlja detaljnije opisujem korake algoritma te najvažnije pomoćne strukture podataka i algoritme. 	