\chapter*{Zaključak}
\addcontentsline{toc}{chapter}{Zaključak}

	Ovaj sustav bio bi od velike pomoći velikom broju fakulteta, škola i brojnim organizatorima programerskih natjecanja jer u relativno kratkom roku može obraditi veliki broj podataka što je i dokazano testovima i uvelike bi smanjio pokušaje pisanja plagijata. Iako rezultati iz literature za deanonimizaciju autora nisu postignuti, opet su bili dovoljno dobri kako bi se mogli koristiti unutar sustava. \\
	
	Duboko vjerujem da bi se ovaj sustav trebao nastaviti razvijati te kao najbitniju stavku proširenja vidim dodavanje novih programskih jezika poput \textit{Pythona} i \textit{Jave} jer se ti jezici se danas najviše koriste za pisanje laboratorijskih vježbi na fakultetima. Dalje kako bi bili otporniji na više modifikacija izvornog koda treba napisat dodatne postupke preoblikovanja koda. Trebalo bi smisliti i nove leksičke, strukturne i sintaksne značajke kako bi deanonimizacija autora radila bolje.  \\
	
	Kao završni cilj $Turtle$ vidim kao sustav koji će fakultetima omogućiti lagano upravljanje laboratorijskim vježbama za niz predmeta koji bi uključivao sve od predaje rješenja do evaluacije istog te detekcije plagijata. 
