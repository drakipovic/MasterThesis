\chapter*{Uvod}
\addcontentsline{toc}{chapter}{Uvod}

	Plagijat(eng. \textit{Plagiarism}) ili postupak krađe tuđeg rada postaje sve veći problem u današnjem svijetu te ga pronalazimo akademskom(npr. eseji, znanstveni članci) i neakademskom(npr. knjige, pjesme) svijetu. Razlog tomu je što količina podataka na internetu, koji je postao dostupan velikom broju ljudi, raste eksponencijalno te je vrlo lagano ukrasti tuđi rad i predstaviti ga kao vlastiti. U ovom radu naglasak će biti na detekciji plagijata izvornih kodova. Cilj je izraditi sustav koji bi ubrzao i uvelike pomogao u detekciji plagijata među izvornim kodovima ljudima koji se brinu o programerskim natjecanjima, laboratorijskim vježbama itd. \\

	Detekciji pristupamo iz dva kuta, jedan je određivanje autora izvornog koda tzv. deanonimizacija autorstva, a drugi određivanje sličnosti među parovima izvornih kodova.   Određivanje sličnosti parova izvornih kodova je već opisano i implementirano u završnom radu autora ovog diplomskog rada, no ovdje predstavljam brže te arhitektualno ljepše rješenje istog problema. Za određivanje autorstva bitno je primjetiti da svaki autor ostavlja svoj jedinstveni otisak dok piše programiski kod, barem se tome nadamo. Kako bi autore razlikovali korištene su tehnike strojnog učenja u konkretnom slučaju klasifikator nasumične šume. Korisnik mora najprije predati skup za treniranje, koji se sastoji od izvornih kodova te točno označenih autora, klasifikatoru kako bi kasnije mogao utvrditi autorstvo na skupu koji želi provjeriti. Korisnika se prepoznaje po njegovom stilu programiranja tako da se izvorni kod pretvori u vektor brojeva od kojih je svaki broj nekakva unaprijed označena značajka(npr. ostavlja li autor novi red prije otvaranja vitičastih zagrada, koliko naredbi grananja koristi, itd.). \\

	Dva pristupa su na kraju integrirana pod istim web sučeljem nazvanim Turtle. Ovo sučelje nudi detaljan uvid u slične parove izvornih kodova te boja slične dijelove jednakim bojama kako bi korisnik prije odlučio promatra li plagijat. 
\newpage
\newgeometry{bottom=25mm, top=25mm, right=25mm, left=30mm}	
\noindent Potrebno je naglasiti da je sustav trenutno implementiran za pronalazak plagijata jedino u programskom jeziku C++. Također ispisuje autore za koje misli da imaju najveću vjerojatnost da su napisali promatrani kod. Utvrđivanje autorstva je veliki korak naprijed nad rješenjem napisanim za završni rad jer nam omogućuje detekciju plagijata među raznim akademskim godinama ukoliko se laboratorijski zadaci mijenjaju.

\newgeometry{bottom=25mm, top=0mm, right=25mm, left=30mm}
