\chapter*{Uvod}
\addcontentsline{toc}{chapter}{Uvod}

	Plagijat (eng. \textit{Plagiarism}) ili postupak krađe tuđeg rada postaje sve veći problem u današnjem svijetu te ga pronalazimo akademskom(npr. eseji, znanstveni članci) i neakademskom(npr. knjige, pjesme) svijetu. Razlog tomu je što količina podataka na internetu, koji je postao dostupan velikom broju ljudi, raste velikom brzinom te je vrlo lagano ukrasti tuđi rad i predstaviti ga kao vlastiti. Ručna potraga za plagijatima na desecima tisuća dokumenata bi trajala vječno te se razvijaju sustavi za detekciju plagijata. Takvi sustavi ne znaju direktno odrediti promatraju li plagijat već najčešće korisniku ispisuju sličnost među parovima dokumenata kako bi korisnik što lakše i brže odlučio. U ovom radu naglasak će biti na detekciji plagijata izvornih kodova. Cilj je izraditi sustav koji bi ubrzao i uvelike pomogao u detekciji plagijata među izvornim kodovima ljudima koji se brinu o programerskim natjecanjima, laboratorijskim vježbama itd. \\

	Detekciji pristupamo iz dva kuta, jedan je određivanje autora izvornog koda tzv. deanonimizacija autorstva, a drugi određivanje sličnosti među parovima izvornih kodova. Određivanje sličnosti među parovima izvornih kodova je problem opisan i implementiran u završnom radu autora \cite{plagijator}, no ovdje je implementirano rješenje koje je i do pet puta brže (skup od 1000 izvornih kodova se izvrši u 23 sekunde) od svoje prve inačice te koje je robusnije na više pokušaja plagiranja izvornog koda kao što je zamjena $for$ u $while$ petlje. Za utvrđivanje autorstva izvornih kodova bitno je primjetiti da svaki autor dok piše izvorni kod ostavlja svoj jedinstveni otisak, ukoliko se ne radi o projektu na kojem postoje točno utvrđena pravila kojih se svi pridržavaju. Ovdje se ipak kreće od pretpostavke da svaki autor ima svoj jedinstveni stil te su korištene tehnike strojnog učenja kako bi ih naučili razlikovati. U konkretnom slučaju korišten je klasifikator slučajne šume. Izvorni kod se pretvara u brojčani vektor značajki. Neke od značajki su frekvencija korištenja ključnih riječi programskog jezika ili koristi li autor više tabove ili razmake na počecima linija. Klasifikator je naučen na dva skupa izvornih kodova preuzetih sa stranica HONI-a(Hrvatsko otvorene natjecanja u informatici). Na manjem skupu od 29 autora postignuta je točnost od 94\% dok je na većem skupu sa 216 autora točnost iznosila 76\%.

	Dva pristupa su na kraju integrirana pod istim web sučeljem nazvanim \textit{Turtle}. \textit{Turtle} je prva, prema autorovom najboljem znanju, web aplikacija koja objedinjuje ova dva pristupa. Ovo sučelje nudi predaju datoteka s izvornim kodovima te uvid u sve parove izvornih kodova za koje sumnja da su plagijati te ispisuje njihove sličnosti. Također kako bi detekcija bila još lakša i brža nudi uvid u izvorne kodove svakog para te boja slične dijelove jednakim bojama. Kako bi koristili utvrđivanje autorstva izvornih kodova najprije moramo sustavu predati datoteku s izvornim kodova na kojima se klasifikator trenira. Nakon što sustav istrenira klasifikator, on se sprema te ga se može kasnije koristiti. Korištenjem istreniranog klasifikatora i predajom izvornih kodova za koje želimo utvrditi autorstvo dobijemo za svaki od predanih izvornih kodova tri autora za koje klasifikator odredi da imaju najveću vjerojatnost biti autori tog izvornog koda. Utvrđivanje autorstva je veliki korak naprijed nad određivanjem sličnosti jer nam omogućuje detekciju plagijata među raznim akademskim godinama ukoliko se laboratorijski zadaci mijenjaju. Za kraj treba naglasiti da sustav trenutno podržava samo jezik $C++$, ali je u implementacija ostavljena mogućnost lakog dodavanja novih jezika.
