\begin{center}
\textbf{Sustav za detekciju plagijata te određivanje autorstva izvornih kodova} \\
\textbf{Sažetak}
\end{center}

	Plagijati postaju sve veći problem u današnjem svijetu, zbog ogromnog rasta broja informacija na internetu vrlo se jednostavno može ukrasti tuđi rad te predstaviti ga kao svoj, tj. plagirati. U sklopu ovog diplomskog rada naglasak je bio na pronalasku plagijata među izvornim kodovima te smo tom problemu pristupili iz dva kuta, prvi je pronalazak sličnosti među parovima izvornih kodova korištenjem otisaka izvornih kodova dok je drugi određivanje autora izvornog koda metodama strojnog učenja. Na kraju su ova dva pristupa spojena u web aplikaciju $Turtle$ koja vrlo uspješno i u kratkom roku pronalazi plagijate među velikim brojem izvornih kodova(npr. laboratorijske vježbe ili programerska natjecanja). \\
	
\noindent \textbf{Ključne riječi:} Strojno učenje, plagijati, deanonimizacija, detekcija plagijata, izvorni kod

\begin{center}
\textbf{System for plagiarism and authorship detection of source code} \\
\textbf{Abstract}
\end{center}

	Plagiarism is becoming bigger and bigger problem in today society, the reason behind it is that information available online is growing exponentialy and it's easy to steal from other authors and present it like your work. This thesis tackles the problem of detecting source code plagiarism from two different approaches, one is computing similiarities between source code using source code fingerprints and other is deanonymizing authors of source code using machine learning. These approaches are then connected to a web application $Turtle$. This application can detect plagiarism between large number of source code(e.g. laboratory assignments, programming competitions) rather well. \\
	
\noindent \textbf{Keywords:} Machine learning, plagiarism, deanonymization, plagiarism detection, source code