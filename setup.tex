\documentclass[12pt,a4paper]{report}
%%%%%%%%%%%%%%%%%%%%%%%%%%%%%%%%%%%%%%%%%%%%%%%%
% Language, Encoding and Fonts
% http://en.wikibooks.org/wiki/LaTeX/Internationalization
%%%%%%%%%%%%%%%%%%%%%%%%%%%%%%%%%%%%%%%%%%%%%%%%
% Select encoding of your inputs. Depends on
% your operating system and its default input
% encoding. Typically, you should use
%   Linux  : utf8 (most modern Linux distributions)
%            latin1 
%   Windows: ansinew
%            latin1 (works in most cases)
%   Mac    : applemac
% Notice that you can manually change the input
% encoding of your files by selecting "save as"
% an select the desired input encoding. 
\usepackage[utf8]{inputenc}
\RequirePackage[croatian]{babel}
\usepackage[sc]{mathpazo}
\linespread{1.5} 
\usepackage{adjustbox}
\usepackage[table,xcdraw]{xcolor}        
\usepackage[T1]{fontenc}
\usepackage{xcolor}
\definecolor{aaublue}{RGB}{33,26,82}
\usepackage{graphicx}
\usepackage{caption}
\usepackage{float}
\usepackage{listings}
\usepackage{multicol,caption}

\lstset{             
  numbers=left,                   % where to put the line-numbers
  stepnumber=1,                   % the step between two line-numbers.        
  numbersep=5pt,                  % how far the line-numbers are from the code
  backgroundcolor=\color{white},  
  showspaces=false,               % show spaces adding particular _
  showstringspaces=false,         % underline spaces within strings
  showtabs=false,                 
  tabsize=2,                      % sets default tabsize to 2 spaces
  captionpos=b,                   
  xleftmargin=\parindent,
  breaklines=true,                % sets automatic line breaking
  breakatwhitespace=true,         
  basicstyle=\footnotesize\ttfamily,
  keywordstyle=\bfseries\color{green!40!black},
  commentstyle=\itshape\color{purple!80!black},
  identifierstyle=\color{blue},
  stringstyle=\color{orange}
}
\captionsetup{%
  font=footnotesize,% set font size to footnotesize
  labelfont=bf % bold label (e.g., Figure 3.2) font
}
% Make the standard latex tables look so much better
\usepackage{array,booktabs}
% Enable the use of frames around, e.g., theorems
% The framed package is used in the example environment
\usepackage{framed}
\usepackage{amsmath}
\usepackage{amssymb}
% Use theorems in your document
% The ntheorem package is also used for the example environment
% When using thmmarks, amsmath must be an option as well. Otherwise \eqref doesn't work anymore.
\usepackage[framed,amsmath,thmmarks]{ntheorem}
\usepackage[
 top = 0mm, bottom = 25mm, inner = 30mm, outer = 25mm]{geometry}
% Modify how \chapter, \section, etc. look
% The titlesec package is very configureable
\usepackage{titlesec}
\titleformat{\chapter}[display]{\normalfont\huge\bfseries}{\filleft\MakeUppercase{\chaptertitlename} \thechapter}{5mm}
{\titlerule
\vspace{5mm}%
\filright}
[\vspace{5mm}%
\titlerule]

\titleformat*{\section}{\normalfont\Large\bfseries}
\titleformat*{\subsection}{\normalfont\large\bfseries}
\titleformat*{\subsubsection}{\normalfont\normalsize\bfseries}
%\titleformat*{\paragraph}{\normalfont\normalsize\bfseries}
%\titleformat*{\subparagraph}{\normalfont\normalsize\bfseries}

% Clear empty pages between chapters
\let\origdoublepage\cleardoublepage
\newcommand{\clearemptydoublepage}{%
  \clearpage
  {\pagestyle{empty}\origdoublepage}%
}
\let\cleardoublepage\clearemptydoublepage

% Change the headers and footers
%\usepackage{fancyhdr}
%\pagestyle{fancy}
%\fancyhf{} %delete everything
%\renewcommand{\headrulewidth}{0pt} %remove the horizontal line in the header
%\fancyhead[LE,LO]{\small\rightmark} %even page - chapter title
%\fancyhead[LO]{\small\nouppercase\rightmark} %uneven page - section title
%\fancyhead[LE,RO]{\thepage} %page number on all pages
% Do not stretch the content of a page. Instead,
% insert white space at the bottom of the page
\raggedbottom
% Enable arithmetics with length. Useful when
% typesetting the layout.
\usepackage{calc}
\usepackage[backend=bibtex,
  bibencoding=utf8, sorting=none
  ]{biblatex}
\addbibresource{bib/literatura}

%%%%%%%%%%%%%%%%%%%%%%%%%%%%%%%%%%%%%%%%%%%%%%%%
% Misc
%%%%%%%%%%%%%%%%%%%%%%%%%%%%%%%%%%%%%%%%%%%%%%%%
% Add bibliography and index to the table of
% contents
\usepackage[nottoc]{tocbibind}
% Add the command \pageref{LastPage} which refers to the
% page number of the last page
\usepackage{lastpage}
% Add todo notes in the margin of the document
\usepackage[
%  disable, %turn off todonotes
  colorinlistoftodos, %enable a coloured square in the list of todos
  textwidth=\marginparwidth, %set the width of the todonotes
  textsize=scriptsize, %size of the text in the todonotes
  ]{todonotes}

\usepackage{hyperref}
\hypersetup{%
	pdfpagelabels=true,%
	plainpages=false,%
	pdfauthor={Author(s)},%
	pdftitle={Title},%
	pdfsubject={Subject},%
	bookmarksnumbered=true,%
	colorlinks=false,%
	citecolor=black,%
	filecolor=black,%
	linkcolor=black,% you should probably change this to black before printing
	urlcolor=black,%
	pdfstartview=FitH%
}
